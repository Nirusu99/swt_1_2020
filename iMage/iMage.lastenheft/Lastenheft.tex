\documentclass[parskip=full]{scrartcl}
\usepackage[utf8]{inputenc} % use utf8 file encoding for TeX sources
\usepackage[T1]{fontenc}    % avoid garbled Unicode text in pdf
\usepackage[german]{babel}  % german hyphenation, quotes, etc
\usepackage{hyperref}       % detailed hyperlink/pdf configuration
\usepackage{graphicx}       % provides commands for including figures
\usepackage{csquotes}       % provides \enquote{} macro for "quotes"
\usepackage[nonumberlist]{glossaries}     % provides glossary commands
\usepackage{enumitem}
\usepackage{tikz}
\usepackage{tikz-uml}
\usepackage{tikzsymbols}

\makenoidxglossaries
%
% % Glossareinträge
%
\newglossaryentry{Applikation}
{
	name=Applikation,
	plural=Applikationen,
	description={oder auch Anwendungssoftware werden Computerprogramme bezeichnet, die genutzt werden, um eine nützliche oder gewünschte nicht systemtechnische Funktionalität zu bearbeiten oder zu unterstützen. (Beispiele für Anwendungsgebiete sind: \enquote{Bildbearbeitung, E-Mail-Programme, Webbrowser, Textverarbeitung, Tabellenkalkulation oder Computerspiele})}
}

\newglossaryentry{iMage}
{
	name=iMage,
	plural=-,
	description={ist eine Anwendungssoftware der Firma Pear Corp und dient dem Anwenden von Kunstfiltern auf bestimmte Bilder}
}

\newglossaryentry{Komprimierung}
{
	name=Komprimierung,
	plural=Komprimierungen,
	description={ist ein Vorgang um Dateigrößen zu verringern}
}

\newglossaryentry{Domain}
{
    name=Domain,
    description={ist vereinfacht gesagt ein Name für einen gewissen Netzwerkbereich (z.B. \enquote{führt google.com zur Google Suchmaschine})}
}

\title{SWT1: Lastenheftvorlage}
\author{Nils Pukropp, 2301588}

\begin{document}

\maketitle

%
% % Hier beginnt die Gliederung des Lastenhefts
%
\section{Zielbestimmung}
Die Firma Pear Corp soll durch das Produkt in die Lage versetzt werden, Bilder automatisiert aus dem Internet herunterzuladen.

\section{Produkteinsatz}
Das Produkt dient der Firma Pear Corp zum Beschaffen von frei verfügbaren Bildern übers Internet.

Zielgruppe: Nutzer der \gls{Applikation} \gls{iMage}

Plattform: Arch, React, Nintendo Switch, Playstation 4, xBox One X

\section{Funktionale Anforderungen}
\begin{itemize}[nosep]
\item[FA10] Das Suchen von Bildern mit Kriterien(Anzahl, Nutzungsrechte, Dateiformat, Dateigröße)
\item[FA20] Das Suchen von Bildern mit integrierter \gls{Komprimierung}
\item[FA30] Das Suchen von Bildern in angegebenen (Sub-)\glspl{Domain}
\item[FA40] Das Anzeigen der über die Sucher geladenen Bilder mit Kriterien(mittlerer Farbwert, Name, Herkunft)
\item[FA50] Die Bilder können lokal oder auf den Pear Crop Zentralservern gespeichert werden.
\end{itemize}

\section{Produktdaten}
\begin{itemize}[nosep]
\item[PD10] Es sind relevante Daten über die Nutzer zu speichern.
\item[PD20] Es sind relevante Daten über die geladenen Bilder zu speichern
\item[PD30] Es sind Bilder lokal oder über den Zentralserver zu speichern
\end{itemize}

\section{Nichtfunktionale Anforderungen}
\begin{itemize}[nosep]
\item[NF10] Die Funktion /FA10/ soll für 500 Bilder maximal 10 Minuten benötigen und selbständig nach einer Suchdauer von einer Stunde abbrechen.
\item[NF20] Die Funktion /FA20/ soll für 500 Bilder maximal 20 Minuten benötigen und selbständig nach einer Suchdauer von zwei Stunden abbrechen.
\item[NF30] Der Zugriff auf den Zentralserver von Pear Corp soll von mindestens einhundert (100) Nutzern gleich-
zeitig erfolgen können und die Dauer des Hochladens der Bilder maximal linear mit der Anzahl der Bilder
wachsen.
\end{itemize}

\section{Systemmodelle}

\subsection{Szenarien}
\begin{flushleft}
    Eine Firma möchte ihre Bildbearbeitungssoftware mit einer Bibliothek an frei verfügbaren Bildern anbieten.
    Durch dieser Applikation könnte diese Firma nun beliebig viele Bilder aus dem Internet laden um diese dann später 
    nach Verfügbarkeit gefiltert manuell auszuwählen und in ihre Software integrierter. \\
    \vspace{1cm}
    Ein anderes Szenario wäre eine Software die ein Bild automatisiert in ein Mosaik aus Bildern umwandelt. Wichtig dafür wäre eine Bibliothek aus nach Farbe sortierten Bildern.
    Mit dieser Applikation würden dann einfach Bilder aus dem Internet geladen, anschließend nach Farbe sortiert und gespeichert werden. Damit wären dann die Bilder für das Mosaik verfügbar.
\end{flushleft}

\subsection{Anwendungsfälle}
\subsubsection{Bilder suchen/anzeigen/speichern}
\begin{tikzpicture}
    \begin{umlpackage}{Applikation}
        \umlsimpleclass[x=3,y=7]{Bilder suchen}
        \umlsimpleclass[x=6,y=5]{Bilder anzeigen}
        \umlsimpleclass[x=3,y=2]{Bilder lokal speichern}
        \umlsimpleclass[x=9,y=2]{Bilder online speichern}

        \umldep[geometry=|-,stereo=depend, pos stereo=0.5]{Bilder anzeigen} {Bilder  suchen}
        \umldep[geometry=|-|,stereo=depend, pos stereo=0.5]{Bilder lokal speichern} {Bilder  anzeigen}
        \umldep[geometry=|-|,stereo=depend, pos stereo=0.5]{Bilder online speichern} {Bilder  anzeigen}
    \end{umlpackage}
    \umlactor[x=-0.5,y=5]{User}

    \umlassoc[geometry=|-]{User} {Bilder suchen}
    \umlassoc[]{User} {Bilder anzeigen}
    \umlassoc[geometry=|-]{User} {Bilder lokal speichern}

    \umlactor[x=12.5,y=5]{Server}
    \path (12.5,5.25) node[above,inner sep=0pt] {\Ofen[2]};

    \umlassoc[geometry=|-]{Server} {Bilder online speichern}
\end{tikzpicture}

Akteure: Nutzer, Server.

Anwendungsfälle: Bildersuche, Bilder anzeigen, Bilder speichern.

Textuelle Beschreibung: Der Nutzer hat die Möglichkeit Bilder aus dem Internet automatisiert zu laden und nach belieben auch direkt komprimiert. Diese geladenen Bilder können dann, nach Kriterien sortiert, vom User angezeigt werden.
Die angezeigten Bilder können nach belieben ausgewählt und nun entweder lokal beim Nutzer oder auf dem Pearl Corp Server gespeichert werden.



%
% % Automatisch generiertes Glossar (Latex zwei mal ausführen um Glossar anzuzeigen)
%
%\glsaddall % das sorgt dafür, dass alles Glossareinträge gedruckt werden, nicht nur die verwendeten. Das sollte nicht nötig sein!
\printnoidxglossaries
Siehe \url{https://en.wikibooks.org/wiki/LaTeX/Glossary}.




\end{document}
